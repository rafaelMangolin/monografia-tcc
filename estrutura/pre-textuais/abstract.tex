% ABSTRACT--------------------------------------------------------------------------------

\begin{resumo}[ABSTRACT]
\begin{SingleSpacing}

% Não altere esta seção do texto--------------------------------------------------------
\imprimirautorcitacao. \imprimirtitleabstract. \imprimirdata. \pageref {LastPage} f. \imprimirprojeto\ – \imprimirprograma, \imprimirinstituicao. \imprimirlocal, \imprimirdata.\\
%---------------------------------------------------------------------------------------


This monograph approaches the problem of food image classification, which is inserted in the area of pattern recognition. The task of image classification was perfomed in this work using convolutional neural networks (CNN), a deep learning technique. CNNs are neural networks of many layers in which the convolution operation are applied. The use of CNNs has been improving the state of the art in the area of image classification since 2012. One of the biggest problems encountered for the use of CNNs in the classification task is the overfitting, that occurs due to a small amount of samples or the depth of the network created. To solve this problem in this work two techniques were used: the data augmentation and the initialization of the weights of the network from a trained network. The proposed neural network in this project is based on CNN alexnet. The database used was formulated with samples taken from the ImageNet and Food-101, containing 16 classes of images. The best result was obtained when applying the techniques, resulting in an accuracy of 74.56\%.

\textbf{Keywords}: Image classification. Convolutional neural network. Deep learning.

\end{SingleSpacing}
\end{resumo}

% OBSERVAÇÕES---------------------------------------------------------------------------
% Altere o texto inserindo o Abstract do seu trabalho.
% Escolha de 3 a 5 palavras ou termos que descrevam bem o seu trabalho 
