% ABSTRACT--------------------------------------------------------------------------------

\begin{resumo}[ABSTRACT]
\begin{SingleSpacing}

% Não altere esta seção do texto--------------------------------------------------------
\imprimirautorcitacao. \imprimirtitleabstract. \imprimirdata. \pageref {LastPage} f. \imprimirprojeto\ – \imprimirprograma, \imprimirinstituicao. \imprimirlocal, \imprimirdata.\\
%---------------------------------------------------------------------------------------


This work addresses the problem of food image classification, which is inserted in the area of pattern recognition. The task of images classification was perfomed in this work using convolutional neural networks (CNN), a deep learning technique. CNN is a neural network of many layers in which the convolution operation is applied. The use of CNN has been improving the state of the art in the area of image classification since 2012. One of the biggest problems encountered for the use of CNN in classification tasks is the overfitting, that occurs due to a small amount of samples or the depth of the network created. To solve this problem, in this work, the fine-tuning was performed at the dropout rate and two techniques were used: the data augmentation and the initialization of the weights of the network from a trained network. The neural network proposed in this project is based on the CNN AlexNet. The database was formulated with samples taken from the ImageNet and Food-101 datasets, containing 16 classes of images. The best result was obtained when applying data augmentation, dropout and weight initialization techniques, resulting in an accuracy of 74.56\%.

\textbf{Keywords}: Image classification. Convolutional neural network. Deep learning.

\end{SingleSpacing}
\end{resumo}

% OBSERVAÇÕES---------------------------------------------------------------------------
% Altere o texto inserindo o Abstract do seu trabalho.
% Escolha de 3 a 5 palavras ou termos que descrevam bem o seu trabalho 
