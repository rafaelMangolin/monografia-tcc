% ABSTRACT--------------------------------------------------------------------------------

\begin{resumo}[ABSTRACT]
\begin{SingleSpacing}

% Não altere esta seção do texto--------------------------------------------------------
\imprimirautorcitacao. \imprimirtitleabstract. \imprimirdata. \pageref {LastPage} f. \imprimirprojeto\ – \imprimirprograma, \imprimirinstituicao. \imprimirlocal, \imprimirdata.\\
%---------------------------------------------------------------------------------------

This monograph approaches the problem of food image classification, which is inserted in the area of pattern recognition. Convolutional neural network (CNN), was used to perform the classification task.  CNNs are deep learing techniques, that use through multi-layers neural networks to accomplish the processes of learning and classification and been improving the state of the art in the area of image classification since 2012. The overfitting it is recurring problem in classifiers that uses CNN, this problem usually occurs due to the lack of samples for performing neural network training. To solve this problem in this work two techniques have been proposed: the data augmentation and the initialization of network weights from a trained network. The proposed neural network in this project is based on CNN AlexNet. The database used was formulated with samples taken from the bases ImageNet and Food-101, containing 16 classes of images. The best result was obtained when applying the proposed techniques, resulting in a accuracy of 74.56 \%.


\textbf{Keywords}: Image classification. Convolutional neural network. Deep learing.

\end{SingleSpacing}
\end{resumo}

% OBSERVAÇÕES---------------------------------------------------------------------------
% Altere o texto inserindo o Abstract do seu trabalho.
% Escolha de 3 a 5 palavras ou termos que descrevam bem o seu trabalho 
