% RESUMO--------------------------------------------------------------------------------

\begin{resumo}[RESUMO]
\begin{SingleSpacing}

% Não altere esta seção do texto--------------------------------------------------------
\imprimirautorcitacao. \imprimirtitulo. \imprimirdata. \pageref {LastPage} f. \imprimirprojeto\ – \imprimirprograma, \imprimirinstituicao. \imprimirlocal, \imprimirdata.\\
%---------------------------------------------------------------------------------------


Esta monografia aborda o problema de classificação de imagens de comida, que está inserido na área de reconhecimento de padrões. A tarefa de classificação de imagem foi realizada neste trabalho utilizando redes neurais convolucionais (\textit{convolutional neural networks}, CNN), uma técnica de \textit{deep learning}. As CNNs são redes neurais de muitas camadas na qual é aplicada a operação de convolução. A utilização de CNNs vem melhorando o estado da arte na área de classificação de imagens desde 2012. Um dos maiores problemas encontrados para o uso de CNNs na tarefa de classificação é o \textit{overfitting}, que ocorre devido a uma quantidade escassa de amostras ou dado a profundidade da rede criada. Para a solução desse problema neste trabalho foram utilizadas duas técnicas: o \textit{data augmentation} e a inicialização dos pesos da rede a partir de uma rede treinada. A rede neural proposta neste projeto tem como base de estrutura a CNN \textit{AlexNet}. A base de dados utilizada foi formulada com amostras retiradas das bases \textit{ImageNet} e \textit{Food-101}, contendo 16 classes de imagens. O melhor resultado foi obtido quando aplicada as técnicas, resultando em uma acurácia de 74,56\%.



% Rede neural convolucional (\textit{convolutional neural network}, CNN) foi utilizada neste trabalho para realizar a tarefa de classificação. As CNNs são técnicas de \textit{deep leaning}, que utilizam por meio de redes neurais de muitas camadas para realizar os processos de aprendizado e classificação e vem melhorando o estado da arte na área de classificação de imagens desde 2012. O \textit{overfitting} é um problema recorrente em classificadores que utilizam CNN, esse problema ocorre geralmente devido a falta de amostras para a realização do treino da rede neural. Para a solução desse problema neste trabalho foi proposto duas técnicas: o \textit{data augmentation} e a inicialização dos pesos da rede a partir de uma rede treinada. A rede neural proposta neste projeto tem como base de estrutura a CNN \textit{AlexNet}. A base de dados utilizada foi formulada com amostras retiradas das bases \textit{ImageNet} e \textit{Food-101}, contendo 16 classes de imagens. O melhor resultado foi obtido quando aplicada as técnicas propostas, resultando em uma acurácia de 74,56\%.

\textbf{Palavras-chave}: Classificação de imagens. Rede neural convolucional. \textit{Deep learing}.

\end{SingleSpacing}
\end{resumo}

% OBSERVAÇÕES---------------------------------------------------------------------------
% Altere o texto inserindo o Resumo do seu trabalho.
% Escolha de 3 a 5 palavras ou termos que descrevam bem o seu trabalho 

