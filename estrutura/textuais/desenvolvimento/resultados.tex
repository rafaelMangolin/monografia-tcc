% RESULTADOS-------------------------------------------------------------------

\chapter{ANÁLISE E DISCUSSÃO DOS RESULTADOS}

% Cada capítulo deve conter uma pequena introdução (tipicamente, um ou dois parágrafos) que deve deixar claro /o objetivo e o que será discutido no capítulo, bem como a organização do capítulo.

Nesta seção está descrito os resultados obtidos a partir dos testes realizados com a rede neural convolucional.
 A métrica utilizada para avaliar os modelos foi a acurácia, que determina a porcentagem de acerto do classificador.%TODO definir melhor a metrica
 Os testes foram iniciados a partir da rede neural convolucional sem nenhuma alteração, seguindo para a inclusão das alterações na base e melhorias na rede, com o fim de avaliar os resultados obtidos e determinar o modelo com melhor acurácia.

\section{Modelo inicial}
O primeiro teste realizado foi com a base de dados sem modificação sobre o modelo inicial de rede neural proposta, com uma configuração de execução com 80 épocas e uma taxa de redução de neurônios nas camadas totalmente conectadas de 50\%. Como visto na \autoref{tab:resultado1} no final da execução da época 80 foi obtido uma acurácia na fase de teste de 53,6\% e 94,02\% na fase de treino, identificando assim um \textit{overfitting} do classificador sobre as amostras da base de treino. É possível perceber dado a análise do \autoref{gra:resultado1} que a acurácia da fase de teste a partir de 42 épocas mantém valores constantes, não apresentando uma ganho expressivo, diferente da acurácia obtida na fase de treino que mantém uma curva crescente.
%TODO colocar tabela e grafico
\section{Modelos com \textit{data augmentation} e inicialicialização dos pesos}
Também foram realizados teste com as estrategias de melhorias da rede descritas anteriormente 
%TODO colocar ref do capitulo de melhorias
, com o objetivo de reduzir o \textit{overfitting} na rede neural. Foram realizados três testes: o teste apenas com a técnica de \textit{data augmentation}; o teste com a técnica de inicialização dos pesos a partir dos valores treinados na rede desenvolvida por \citeonline{imaginetArticle}; e o teste aplicando as duas técnicas. 

\par Como informado na \autoref{tab:resultado2} as acurácias obtidas na fase de teste e treino, respectivamente, apenas com \textit{data augmentation} foram de 58,23\% e 71,66\%, vendo com essa mudança uma redução significativa do \textit{overfitting} na rede, além de uma melhora da acurácia na fase de teste.
%TODO inserir aki a tabela e o grafico
\par O teste realizado com a inicialização dos pesos obteve um resultado mais expressivo se comparado com o teste apenas com \textit{data augmentation} obtendo as acurácias na fase de teste e treino, respectivamente, de 68,08\% e 99,63\%. Com a aplicação dessa técnica foi obtido uma melhora significativa na acurácia se comparado com o modelo inicial e o modelo com \textit{data augmentation}, mas assim como no modelo inicial esse modelo apresenta \textit{overfitting} na fase de treino, como é possível observar no \autoref{gra:resultado2} a partir da época 40 a acurácia de teste permanece estável e já a acurácia da fase de treino continua crescendo até atingir valores próximos a 100\%.

\par Com o intuito de solucionar o problema de \textit{overfitting} e obter melhoria na classificação, foi realizado o teste utilizando as duas técnicas. A acurácia obtida na fase de treino foi de 98,18\% e na fase de teste obteve um valor de 71,77\%, apresentando o melhor resultado entre os três testes realizados.% TODO colocar referencias que apresentam melhoria em modelos aplicados  data augmentantion e inicialização de pesos

 
\section{Aprimoramento da taxa de \textit{Dropout}}
Foram realizados testes com a taxa aplicada na função de \textit{Dropout} buscando diminuir o \textit{overfitting} na rede. A taxa de \textit{dropout} foi variada de 50\% a 90\%, com intervalos de 10\% entre cada teste. Como pode ser verificado na \autoref{tab:resultado3} a taxa de 80\% foi a que apresentou melhor resultado com uma acurácia de 74,56\% na fase de teste e 93,96\% de acurácia na fase de treino.
%TODO inserir aki o grafico e a tabela
\par A partir da \autoref{tab:resultado3} é possível concluir que os teste realizado com a taxa de \textit{dropout} 80\% apresentou o melhor resultado pois se encontra no limiar entre uma taxa que permite aprender características importantes das classes avaliadas sem ficar super adaptada a base de dados de treino (\textit{overfitting}). 

%O teste realizado com a técnica de \textit{data augmentation}, obteve uma melhora no quesito de evitar o \textit{overfitting} do modelo, na qual conforme apresentado na \autoref{tab:resultado2} a acurácia na fase de treino diminuiu para 71,66\%, mas a acurácia obtida na fase de teste aumentou para 58,23\%. Também é possível notar que a acurácia na fase de teste atinge um valor estável a partir de 64 épocas, mostrando  
