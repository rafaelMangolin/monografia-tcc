% CONCLUSÃO--------------------------------------------------------------------

\chapter{CONCLUSÃO}
\label{chap:conclusao}

A tarefa de classificação de imagem vem sendo um dos grandes desafios na área de visão computacional. Utilizar \textit{deep learning}, com CNN, para a classificação de imagens é algo que ocorre desde 2012 e vem atualizando o estado da arte nessa área. Neste trabalho utilizamos de CNN para realizar a classificação de imagens de comida, utilizando técnicas de \textit{data augmentation} e inicialização dos pesos da rede para obter uma melhor classificação e reduzir o \textit{overfitting}, uns dos grandes problemas no uso de CNN.
\par O \textit{overfitting} é um dos grandes obstáculos para o uso de CNNs. Uma base de treino muito pequena ou uma rede neural muito profunda~são possíveis causas do \textit{overfitting}. A aplicação de técnicas para aumento da base de dados e a utilização do \textit{dropout} em algumas camadas da rede são medidas que reduzem o \textit{overfitting}, conforme foi aplicado e constatado neste trabalho.
\par \textit{Data augmentation} vem sendo aplicado nos modelos de classificação que utilizam CNN. A aplicação do \textit{data augmentation} na base foi feita com o objetivo de modificar ligeiramente as imagens originais. Também foi utilizado para melhorar o resultado de classificação a inicialização dos pesos da rede com os valores obtidos a partir de um rede já treinada, ao invés de realizar a inicialização com dados aleatórios.   
\par Nos experimentos realizados o emprego da técnica de \textit{data augmentation}, como relatado por outros trabalhos, obteve um resultado expressivo para realizar a redução do \textit{overfitting}, além que alcançar uma melhora na acurácia. A aplicação em conjunto do \textit{data augmentation}, inicialização dos pesos e otimização da taxa de \textit{dropout} obteve um resultado de 74,56\% de acurácia, aumentando em 20,96\% a acurácia se comparado com o modelo proposto sem as melhorias (53,6\% de acurácia).
\par Outra abordagem que pode ser realizada futuramente é utilizar o modelo para aplicar a extração das características em conjunto com outro classificador como SVM (do inglês \textit{suport vector machine}) para realizar a predição. A aplicação de vetores de dissimilaridade nas características extraídas com a CNN para a classificação seria outro caminho a ser abordado. Otimizações de outros parâmetros da rede como a taxa de aprendizagem e o estudo de outras arquiteturas de CNN como a \textit{GoogLeNet} \cite{szegedy2015going} para melhorar o desempenho do classificador. A aplicação de técnicas para redução dos parâmetros, possibilitaria o uso desse classificador em dispositivos com uma menor capacidade de processamento.