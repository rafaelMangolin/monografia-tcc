% INTRODUÇÃO-------------------------------------------------------------------

\chapter{INTRODUÇÃO}
\label{chap:introducao}
Segundo \cite{lecun2015deep}, o \textit{deep learning} vem sendo utilizado para resolver problemas computacionalmente complexos que temos no nosso dia-a-dia, e seu uso vem evoluindo o estado-da-arte de muitas áreas. Este projeto será aplicado na área de reconhecimento de imagem, para classificação de imagens.

\par Neste projeto iremos utilizar de \textit{deep learning}, mais especificamente, de rede neural convolucional, para fazer o reconhecimento e classificação de imagens de comida, visando reconhecer o tipo de comida descrito na imagem, a partir de um conjunto de tipos previamente definidos.
\par Na seção de fundamentação teórica desse projeto são descritos conceitos que auxiliam no seu desenvolvimento, sendo dividida em dois tópicos principais: Redes Neurais e \textit{Deep learning}. Também possui as seções de motivação e objetivos (geral e específicos), justificando o desenvolvimento desse trabalho, bem como definindo quais os objetivos a serem completados no seu término.
\par Na seção de materiais e métodos é explicado quais materiais de apoio serão necessário  para a execução do projeto. E na seção de cronograma são descritos as tarefas que devem ser cumpridas, bem como um período para a sua execução.