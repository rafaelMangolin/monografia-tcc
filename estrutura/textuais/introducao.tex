% INTRODUÇÃO-------------------------------------------------------------------

\chapter{INTRODUÇÃO}
\label{chap:introducao}
Nos dias de hoje a interação das pessoas por meios de redes sociais se tornou parte do cotidiano. 
Imagens são compartilhadas nas redes sociais para ilustrar fatos que ocorrem no dia a dia, como refeições realizadas ou locais visitados, se tornando um hábito cada vez mais comum. Dessa maneira, as redes sociais se tornaram um repositório vasto de imagens de diversos conteúdos que descrevem características e gostos dos seus usuários. A classificação de imagens aparece como solução para realizar a categorização automática dessas imagens \cite{imaginetArticle}.

\par O problema de classificação de imagens utilizando técnicas tradicionais de aprendizado de máquina requer conhecimento e experiência no domínio que será classificado, para extrair características (do inglês \textit{features}) relevantes, que são utilizadas para obter a classificação. Assim, utilizando técnicas que não dependem de extratores especializados de \textit{features}, como redes neurais convolucionais (CNN, \textit{convolutional neural networks}), se torna mais fácil desenvolver modelos eficazes de aprendizado de máquina para novos conjuntos de dados.

\par Segundo \citeonline{lecun2015deep}, o \textit{deep learning} vem sendo utilizado para resolver problemas computacionalmente complexos que temos no nosso dia-a-dia, e seu uso vem evoluindo o estado-da-arte de muitas áreas.

\par Um dos grandes problemas encontrados no uso de CNN é o \textit{overfitting} \cite{imaginetArticle}, que ocorre devido a muitos parâmetros na rede ou uma base de dados pequena. A utilização de técnicas como o \textit{dropout} e o \textit{data augmentation} vem sendo utilizas para reduzir o \textit{overfitting} \cite{srivastava2014dropout}.

\par Neste projeto foi utilizada CNN para a classificação de imagens com comida, assim não será preciso encontrar descritores especializados nesse domínio. A arquitetura da rede neural desenvolvida foi baseada na CNN \textit{AlexNet} \citeonline{imaginetArticle}. Foram aplicadas na arquitetura da rede a operação de \textit{dropout} e a técnica de \textit{data augmentation} na base de dados visando reduzir o \textit{overfitting}.


\par Este projeto tem como objetivo realizar o reconhecimento e classificação de imagens de comida, visando reconhecer o tipo de comida descrito na imagem, a partir de um conjunto de tipos previamente definidos. Os experimentos realizados no modelo proposto utilizaram uma base de dados formulada a partir das bases de imagens \textit{ImageNet}\cite{deng2009imagenet} e \textit{Food-101}\cite{bossard14}. Com a execução desses experimentos foi possível avaliar o modelo, com a variação das diferentes técnicas propostas, conseguindo comparar os resultados e identificar o melhor modelo obtido. 

\par O modelo inicial da rede neural sem a aplicação de melhorias obteve um resultado de $53,6\%$ de acurácia. O modelo com \textit{data augmentation} e inicialização dos pesos gerou uma acurácia de $71,77\%$, mostrando a melhoria com o uso das técnicas, mas ainda apresentando um alto \textit{overfitting}. Com o ajuste da taxa de \textit{dropout} foi obtido o melhor resultado, apresentando um redução do \textit{overfitting} e obtendo um acurácia de $ 74,56\%$.

%\par Neste projeto utilizamos de \textit{deep learning}, mais especificamente, de rede neural convolucional, para fazer o reconhecimento e classificação de imagens de comida, visando reconhecer o tipo de comida descrito na imagem, a partir de um conjunto de tipos previamente definidos. Foi aplicada na arquitetura da rede a operação de \textit{dropout} e a técnica de \textit{data augmentation} na base de dados visando reduzir o \textit{overfitting}. 

\par No \autoref{chap:fundamentacaoTeorica} de revisão de literatura desse projeto são descritos conceitos que auxiliam no seu desenvolvimento, sendo dividida em dois tópicos principais: Redes Neurais e \textit{Deep learning}. O \autoref{chap:metodologia} de metodologia descreve como foi estruturado o projeto, e as etapas utilizadas na sua aplicação, composta por três tópicos principais: a definição e organização da base de dados; a arquitetura da rede neural proposta; e técnicas para melhorar o poder de classificação da rede neural.
\par O \autoref{chap:resultados} de resultados apresenta os valores obtidos com os experimentos realizados, contendo uma comparação entre os resultados e avaliando as melhorias obtidas conforme as técnicas eram adicionadas ao modelo. A melhora na acurácia e a redução do \textit{overfitting} são pontos discutidos. No \autoref{chap:conclusao} de conclusão é apresentada uma síntese sobre o modelo e os resultados obtidos, também é proposta atividades para a continuação deste trabalho.
