% INTRODUÇÃO-------------------------------------------------------------------

\chapter{INTRODUÇÃO}
\label{chap:introducao}
O problema de classificação de imagens utilizando técnicas tradicionais de aprendizado de máquina requer conhecimento e experiência no domínio que será classificado, para extrair características (do inglês \textit{features}) relevantes, que são utilizadas para a obter a classificação. Assim, utilizando técnicas que não dependem de extratores especializados de \textit{features}, como redes neurais convolucionais (CNN, \textit{convolutional neural networks}), se torna mais fácil desenvolver modelos eficazes de aprendizado de máquina para novos conjuntos de dados. Nesse projeto será utilizada CNN para a classificação de imagens com comida, assim não será preciso encontrar descritores especializados nesse domínio. 
\par Segundo \cite{lecun2015deep}, o \textit{deep learning} vem sendo utilizado para resolver problemas computacionalmente complexos que temos no nosso dia-a-dia, e seu uso vem evoluindo o estado-da-arte de muitas áreas. Este projeto será aplicado na área de reconhecimento de imagem, para classificação de imagens.
\par Neste projeto utilizamos de \textit{deep learning}, mais especificamente, de rede neural convolucional, para fazer o reconhecimento e classificação de imagens de comida, visando reconhecer o tipo de comida descrito na imagem, a partir de um conjunto de tipos previamente definidos.
\par Na seção de revisão de literatura desse projeto são descritos conceitos que auxiliam no seu desenvolvimento, sendo dividida em dois tópicos principais: Redes Neurais e \textit{Deep learning}. Na seção de metodologia onde está definida a estrutura e organização da base de dados, descrita a estrutura da rede neural, bem como tecnincas para melhorar sua performance de classificação. Já na seção de resultados é avaliado os resultados obtidos com os testes realizados


% motivação e objetivos (geral e específicos), justificando o desenvolvimento desse trabalho, bem como definindo quais os objetivos a serem completados no seu término.
% \par Na seção de materiais e métodos é explicado quais materiais de apoio serão necessário  para a execução do projeto. E na seção de cronograma são descritos as tarefas que devem ser cumpridas, bem como um período para a sua execução.