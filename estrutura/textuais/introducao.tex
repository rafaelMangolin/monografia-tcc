% INTRODUÇÃO-------------------------------------------------------------------

\chapter{INTRODUÇÃO}
\label{chap:introducao}
O problema de classificação de imagens utilizando técnicas tradicionais de aprendizado de máquina requer conhecimento e experiência no domínio que será classificado, para extrair características (do inglês \textit{features}) relevantes, que são utilizadas para a obter a classificação. Assim, utilizando técnicas que não dependem de extratores especializados de \textit{features}, como redes neurais convolucionais (CNN, \textit{convolutional neural networks}), se torna mais fácil desenvolver modelos eficazes de aprendizado de máquina para novos conjuntos de dados. Nesse projeto foi utilizada CNN para a classificação de imagens com comida, assim não será preciso encontrar descritores especializados nesse domínio. 
\par Segundo \cite{lecun2015deep}, o \textit{deep learning} vem sendo utilizado para resolver problemas computacionalmente complexos que temos no nosso dia-a-dia, e seu uso vem evoluindo o estado-da-arte de muitas áreas. Este projeto foi aplicado na área de reconhecimento de imagem, para classificação de imagens.
\par Neste projeto utilizamos de \textit{deep learning}, mais especificamente, de rede neural convolucional, para fazer o reconhecimento e classificação de imagens de comida, visando reconhecer o tipo de comida descrito na imagem, a partir de um conjunto de tipos previamente definidos.
\par Na seção de revisão de literatura desse projeto são descritos conceitos que auxiliam no seu desenvolvimento, sendo dividida em dois tópicos principais: Redes Neurais e \textit{Deep learning}. Na seção de metodologia descreve como foi descrito o projeto, e as etapas utilizadas na sua aplicação, é composta por três tópicos principais: a definição e organização da base de dados; a arquitetura da rede neural proposta; e técnicas para melhorar o poder de classificação da rede neural.
\par A seção de resultados contém uma avaliação dos resultados obtidos, contendo uma comparação entre os resultados e avaliando a melhora na acurácia e a redução do \textit{overfitting}, conforme as técnicas eram adicionadas ao modelo. Na seção de conclusão é apresentada as contribuições deste trabalho na área e proposto projetos visando a melhora de resultado da rede.
