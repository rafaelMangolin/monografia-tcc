% INTRODUÇÃO-------------------------------------------------------------------

\chapter{INTRODUÇÃO}
\label{chap:introducao}
O problema de classificação de imagens utilizando técnicas tradicionais de aprendizado de máquina requer conhecimento e experiência no domínio que será classificado, para extrair características (do inglês \textit{features}) relevantes, que são utilizadas para obter a classificação. Assim, utilizando técnicas que não dependem de extratores especializados de \textit{features}, como redes neurais convolucionais (CNN, \textit{convolutional neural networks}), se torna mais fácil desenvolver modelos eficazes de aprendizado de máquina para novos conjuntos de dados. Nesse projeto foi utilizada CNN para a classificação de imagens com comida, assim não será preciso encontrar descritores especializados nesse domínio. 
\par Segundo \citeonline{lecun2015deep}, o \textit{deep learning} vem sendo utilizado para resolver problemas computacionalmente complexos que temos no nosso dia-a-dia, e seu uso vem evoluindo o estado-da-arte de muitas áreas.
\par Um dos grandes problemas encontrados no uso de CNN é o \textit{overfitting} \cite{imaginetArticle} que ocorre devido a muitos parâmetros na rede ou uma base de dados pequena. A utilização de técnicas como o \textit{dropout} e o \textit{data augmentation} vem sendo utilizas para reduzir o \textit{overfitting} \cite{srivastava2014dropout}.

\par Neste projeto utilizamos de \textit{deep learning}, mais especificamente, de rede neural convolucional, para fazer o reconhecimento e classificação de imagens de comida, visando reconhecer o tipo de comida descrito na imagem, a partir de um conjunto de tipos previamente definidos. Foi aplicada na arquitetura da rede a operação de \textit{dropout} e a técnica de \textit{data augmentation} na base de dados visando reduzir o \textit{overfitting}. 
\par No \autoref{chap:fundamentacaoTeorica} de revisão de literatura desse projeto são descritos conceitos que auxiliam no seu desenvolvimento, sendo dividida em dois tópicos principais: Redes Neurais e \textit{Deep learning}. O \autoref{chap:metodologia} de metodologia descreve como foi estruturado o projeto, e as etapas utilizadas na sua aplicação, composta por três tópicos principais: a definição e organização da base de dados; a arquitetura da rede neural proposta; e técnicas para melhorar o poder de classificação da rede neural.
\par O \autoref{chap:resultados} de resultados apresenta os valores obtidos com os experimentos realizados, contendo uma comparação entre os resultados e avaliando as melhorias obtidas conforme as técnicas eram adicionadas ao modelo. A melhora na acurácia e a redução do \textit{overfitting} são pontos discutidos. No \autoref{chap:conclusao} de conclusão é apresentada uma síntese sobre o modelo e os resultados obtidos, também é proposta atividades para a continuação deste trabalho.
